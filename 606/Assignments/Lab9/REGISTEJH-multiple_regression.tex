% Options for packages loaded elsewhere
\PassOptionsToPackage{unicode}{hyperref}
\PassOptionsToPackage{hyphens}{url}
%
\documentclass[
]{article}
\usepackage{lmodern}
\usepackage{amssymb,amsmath}
\usepackage{ifxetex,ifluatex}
\ifnum 0\ifxetex 1\fi\ifluatex 1\fi=0 % if pdftex
  \usepackage[T1]{fontenc}
  \usepackage[utf8]{inputenc}
  \usepackage{textcomp} % provide euro and other symbols
\else % if luatex or xetex
  \usepackage{unicode-math}
  \defaultfontfeatures{Scale=MatchLowercase}
  \defaultfontfeatures[\rmfamily]{Ligatures=TeX,Scale=1}
\fi
% Use upquote if available, for straight quotes in verbatim environments
\IfFileExists{upquote.sty}{\usepackage{upquote}}{}
\IfFileExists{microtype.sty}{% use microtype if available
  \usepackage[]{microtype}
  \UseMicrotypeSet[protrusion]{basicmath} % disable protrusion for tt fonts
}{}
\makeatletter
\@ifundefined{KOMAClassName}{% if non-KOMA class
  \IfFileExists{parskip.sty}{%
    \usepackage{parskip}
  }{% else
    \setlength{\parindent}{0pt}
    \setlength{\parskip}{6pt plus 2pt minus 1pt}}
}{% if KOMA class
  \KOMAoptions{parskip=half}}
\makeatother
\usepackage{xcolor}
\IfFileExists{xurl.sty}{\usepackage{xurl}}{} % add URL line breaks if available
\IfFileExists{bookmark.sty}{\usepackage{bookmark}}{\usepackage{hyperref}}
\hypersetup{
  pdftitle={Multiple linear regression},
  pdfauthor={Joshua Registe},
  hidelinks,
  pdfcreator={LaTeX via pandoc}}
\urlstyle{same} % disable monospaced font for URLs
\usepackage[margin=1in]{geometry}
\usepackage{color}
\usepackage{fancyvrb}
\newcommand{\VerbBar}{|}
\newcommand{\VERB}{\Verb[commandchars=\\\{\}]}
\DefineVerbatimEnvironment{Highlighting}{Verbatim}{commandchars=\\\{\}}
% Add ',fontsize=\small' for more characters per line
\usepackage{framed}
\definecolor{shadecolor}{RGB}{248,248,248}
\newenvironment{Shaded}{\begin{snugshade}}{\end{snugshade}}
\newcommand{\AlertTok}[1]{\textcolor[rgb]{0.94,0.16,0.16}{#1}}
\newcommand{\AnnotationTok}[1]{\textcolor[rgb]{0.56,0.35,0.01}{\textbf{\textit{#1}}}}
\newcommand{\AttributeTok}[1]{\textcolor[rgb]{0.77,0.63,0.00}{#1}}
\newcommand{\BaseNTok}[1]{\textcolor[rgb]{0.00,0.00,0.81}{#1}}
\newcommand{\BuiltInTok}[1]{#1}
\newcommand{\CharTok}[1]{\textcolor[rgb]{0.31,0.60,0.02}{#1}}
\newcommand{\CommentTok}[1]{\textcolor[rgb]{0.56,0.35,0.01}{\textit{#1}}}
\newcommand{\CommentVarTok}[1]{\textcolor[rgb]{0.56,0.35,0.01}{\textbf{\textit{#1}}}}
\newcommand{\ConstantTok}[1]{\textcolor[rgb]{0.00,0.00,0.00}{#1}}
\newcommand{\ControlFlowTok}[1]{\textcolor[rgb]{0.13,0.29,0.53}{\textbf{#1}}}
\newcommand{\DataTypeTok}[1]{\textcolor[rgb]{0.13,0.29,0.53}{#1}}
\newcommand{\DecValTok}[1]{\textcolor[rgb]{0.00,0.00,0.81}{#1}}
\newcommand{\DocumentationTok}[1]{\textcolor[rgb]{0.56,0.35,0.01}{\textbf{\textit{#1}}}}
\newcommand{\ErrorTok}[1]{\textcolor[rgb]{0.64,0.00,0.00}{\textbf{#1}}}
\newcommand{\ExtensionTok}[1]{#1}
\newcommand{\FloatTok}[1]{\textcolor[rgb]{0.00,0.00,0.81}{#1}}
\newcommand{\FunctionTok}[1]{\textcolor[rgb]{0.00,0.00,0.00}{#1}}
\newcommand{\ImportTok}[1]{#1}
\newcommand{\InformationTok}[1]{\textcolor[rgb]{0.56,0.35,0.01}{\textbf{\textit{#1}}}}
\newcommand{\KeywordTok}[1]{\textcolor[rgb]{0.13,0.29,0.53}{\textbf{#1}}}
\newcommand{\NormalTok}[1]{#1}
\newcommand{\OperatorTok}[1]{\textcolor[rgb]{0.81,0.36,0.00}{\textbf{#1}}}
\newcommand{\OtherTok}[1]{\textcolor[rgb]{0.56,0.35,0.01}{#1}}
\newcommand{\PreprocessorTok}[1]{\textcolor[rgb]{0.56,0.35,0.01}{\textit{#1}}}
\newcommand{\RegionMarkerTok}[1]{#1}
\newcommand{\SpecialCharTok}[1]{\textcolor[rgb]{0.00,0.00,0.00}{#1}}
\newcommand{\SpecialStringTok}[1]{\textcolor[rgb]{0.31,0.60,0.02}{#1}}
\newcommand{\StringTok}[1]{\textcolor[rgb]{0.31,0.60,0.02}{#1}}
\newcommand{\VariableTok}[1]{\textcolor[rgb]{0.00,0.00,0.00}{#1}}
\newcommand{\VerbatimStringTok}[1]{\textcolor[rgb]{0.31,0.60,0.02}{#1}}
\newcommand{\WarningTok}[1]{\textcolor[rgb]{0.56,0.35,0.01}{\textbf{\textit{#1}}}}
\usepackage{longtable,booktabs}
% Correct order of tables after \paragraph or \subparagraph
\usepackage{etoolbox}
\makeatletter
\patchcmd\longtable{\par}{\if@noskipsec\mbox{}\fi\par}{}{}
\makeatother
% Allow footnotes in longtable head/foot
\IfFileExists{footnotehyper.sty}{\usepackage{footnotehyper}}{\usepackage{footnote}}
\makesavenoteenv{longtable}
\usepackage{graphicx,grffile}
\makeatletter
\def\maxwidth{\ifdim\Gin@nat@width>\linewidth\linewidth\else\Gin@nat@width\fi}
\def\maxheight{\ifdim\Gin@nat@height>\textheight\textheight\else\Gin@nat@height\fi}
\makeatother
% Scale images if necessary, so that they will not overflow the page
% margins by default, and it is still possible to overwrite the defaults
% using explicit options in \includegraphics[width, height, ...]{}
\setkeys{Gin}{width=\maxwidth,height=\maxheight,keepaspectratio}
% Set default figure placement to htbp
\makeatletter
\def\fps@figure{htbp}
\makeatother
\setlength{\emergencystretch}{3em} % prevent overfull lines
\providecommand{\tightlist}{%
  \setlength{\itemsep}{0pt}\setlength{\parskip}{0pt}}
\setcounter{secnumdepth}{-\maxdimen} % remove section numbering

\title{Multiple linear regression}
\author{Joshua Registe}
\date{}

\begin{document}
\maketitle

\hypertarget{grading-the-professor}{%
\subsection{Grading the professor}\label{grading-the-professor}}

Many college courses conclude by giving students the opportunity to
evaluate the course and the instructor anonymously. However, the use of
these student evaluations as an indicator of course quality and teaching
effectiveness is often criticized because these measures may reflect the
influence of non-teaching related characteristics, such as the physical
appearance of the instructor. The article titled, ``Beauty in the
classroom: instructors' pulchritude and putative pedagogical
productivity'' (Hamermesh and Parker, 2005) found that instructors who
are viewed to be better looking receive higher instructional ratings.
(Daniel S. Hamermesh, Amy Parker, Beauty in the classroom: instructors
pulchritude and putative pedagogical productivity, \emph{Economics of
Education Review}, Volume 24, Issue 4, August 2005, Pages 369-376, ISSN
0272-7757, 10.1016/j.econedurev.2004.07.013.
\url{http://www.sciencedirect.com/science/article/pii/S0272775704001165}.)

In this lab we will analyze the data from this study in order to learn
what goes into a positive professor evaluation.

\hypertarget{the-data}{%
\subsection{The data}\label{the-data}}

The data were gathered from end of semester student evaluations for a
large sample of professors from the University of Texas at Austin. In
addition, six students rated the professors' physical appearance. (This
is aslightly modified version of the original data set that was released
as part of the replication data for \emph{Data Analysis Using Regression
and Multilevel/Hierarchical Models} (Gelman and Hill, 2007).) The result
is a data frame where each row contains a different course and columns
represent variables about the courses and professors.

\begin{Shaded}
\begin{Highlighting}[]
\KeywordTok{load}\NormalTok{(}\StringTok{"more/evals.RData"}\NormalTok{)}
\end{Highlighting}
\end{Shaded}

\begin{longtable}[]{@{}ll@{}}
\toprule
\begin{minipage}[b]{0.56\columnwidth}\raggedright
variable\strut
\end{minipage} & \begin{minipage}[b]{0.38\columnwidth}\raggedright
description\strut
\end{minipage}\tabularnewline
\midrule
\endhead
\begin{minipage}[t]{0.56\columnwidth}\raggedright
\texttt{score}\strut
\end{minipage} & \begin{minipage}[t]{0.38\columnwidth}\raggedright
average professor evaluation score: (1) very unsatisfactory - (5)
excellent.\strut
\end{minipage}\tabularnewline
\begin{minipage}[t]{0.56\columnwidth}\raggedright
\texttt{rank}\strut
\end{minipage} & \begin{minipage}[t]{0.38\columnwidth}\raggedright
rank of professor: teaching, tenure track, tenured.\strut
\end{minipage}\tabularnewline
\begin{minipage}[t]{0.56\columnwidth}\raggedright
\texttt{ethnicity}\strut
\end{minipage} & \begin{minipage}[t]{0.38\columnwidth}\raggedright
ethnicity of professor: not minority, minority.\strut
\end{minipage}\tabularnewline
\begin{minipage}[t]{0.56\columnwidth}\raggedright
\texttt{gender}\strut
\end{minipage} & \begin{minipage}[t]{0.38\columnwidth}\raggedright
gender of professor: female, male.\strut
\end{minipage}\tabularnewline
\begin{minipage}[t]{0.56\columnwidth}\raggedright
\texttt{language}\strut
\end{minipage} & \begin{minipage}[t]{0.38\columnwidth}\raggedright
language of school where professor received education: english or
non-english.\strut
\end{minipage}\tabularnewline
\begin{minipage}[t]{0.56\columnwidth}\raggedright
\texttt{age}\strut
\end{minipage} & \begin{minipage}[t]{0.38\columnwidth}\raggedright
age of professor.\strut
\end{minipage}\tabularnewline
\begin{minipage}[t]{0.56\columnwidth}\raggedright
\texttt{cls\_perc\_eval}\strut
\end{minipage} & \begin{minipage}[t]{0.38\columnwidth}\raggedright
percent of students in class who completed evaluation.\strut
\end{minipage}\tabularnewline
\begin{minipage}[t]{0.56\columnwidth}\raggedright
\texttt{cls\_did\_eval}\strut
\end{minipage} & \begin{minipage}[t]{0.38\columnwidth}\raggedright
number of students in class who completed evaluation.\strut
\end{minipage}\tabularnewline
\begin{minipage}[t]{0.56\columnwidth}\raggedright
\texttt{cls\_students}\strut
\end{minipage} & \begin{minipage}[t]{0.38\columnwidth}\raggedright
total number of students in class.\strut
\end{minipage}\tabularnewline
\begin{minipage}[t]{0.56\columnwidth}\raggedright
\texttt{cls\_level}\strut
\end{minipage} & \begin{minipage}[t]{0.38\columnwidth}\raggedright
class level: lower, upper.\strut
\end{minipage}\tabularnewline
\begin{minipage}[t]{0.56\columnwidth}\raggedright
\texttt{cls\_profs}\strut
\end{minipage} & \begin{minipage}[t]{0.38\columnwidth}\raggedright
number of professors teaching sections in course in sample: single,
multiple.\strut
\end{minipage}\tabularnewline
\begin{minipage}[t]{0.56\columnwidth}\raggedright
\texttt{cls\_credits}\strut
\end{minipage} & \begin{minipage}[t]{0.38\columnwidth}\raggedright
number of credits of class: one credit (lab, PE, etc.), multi
credit.\strut
\end{minipage}\tabularnewline
\begin{minipage}[t]{0.56\columnwidth}\raggedright
\texttt{bty\_f1lower}\strut
\end{minipage} & \begin{minipage}[t]{0.38\columnwidth}\raggedright
beauty rating of professor from lower level female: (1) lowest - (10)
highest.\strut
\end{minipage}\tabularnewline
\begin{minipage}[t]{0.56\columnwidth}\raggedright
\texttt{bty\_f1upper}\strut
\end{minipage} & \begin{minipage}[t]{0.38\columnwidth}\raggedright
beauty rating of professor from upper level female: (1) lowest - (10)
highest.\strut
\end{minipage}\tabularnewline
\begin{minipage}[t]{0.56\columnwidth}\raggedright
\texttt{bty\_f2upper}\strut
\end{minipage} & \begin{minipage}[t]{0.38\columnwidth}\raggedright
beauty rating of professor from second upper level female: (1) lowest -
(10) highest.\strut
\end{minipage}\tabularnewline
\begin{minipage}[t]{0.56\columnwidth}\raggedright
\texttt{bty\_m1lower}\strut
\end{minipage} & \begin{minipage}[t]{0.38\columnwidth}\raggedright
beauty rating of professor from lower level male: (1) lowest - (10)
highest.\strut
\end{minipage}\tabularnewline
\begin{minipage}[t]{0.56\columnwidth}\raggedright
\texttt{bty\_m1upper}\strut
\end{minipage} & \begin{minipage}[t]{0.38\columnwidth}\raggedright
beauty rating of professor from upper level male: (1) lowest - (10)
highest.\strut
\end{minipage}\tabularnewline
\begin{minipage}[t]{0.56\columnwidth}\raggedright
\texttt{bty\_m2upper}\strut
\end{minipage} & \begin{minipage}[t]{0.38\columnwidth}\raggedright
beauty rating of professor from second upper level male: (1) lowest -
(10) highest.\strut
\end{minipage}\tabularnewline
\begin{minipage}[t]{0.56\columnwidth}\raggedright
\texttt{bty\_avg}\strut
\end{minipage} & \begin{minipage}[t]{0.38\columnwidth}\raggedright
average beauty rating of professor.\strut
\end{minipage}\tabularnewline
\begin{minipage}[t]{0.56\columnwidth}\raggedright
\texttt{pic\_outfit}\strut
\end{minipage} & \begin{minipage}[t]{0.38\columnwidth}\raggedright
outfit of professor in picture: not formal, formal.\strut
\end{minipage}\tabularnewline
\begin{minipage}[t]{0.56\columnwidth}\raggedright
\texttt{pic\_color}\strut
\end{minipage} & \begin{minipage}[t]{0.38\columnwidth}\raggedright
color of professor's picture: color, black \& white.\strut
\end{minipage}\tabularnewline
\bottomrule
\end{longtable}

\hypertarget{exploring-the-data}{%
\subsection{Exploring the data}\label{exploring-the-data}}

\begin{enumerate}
\def\labelenumi{\arabic{enumi}.}
\item
  Is this an observational study or an experiment? The original research
  question posed in the paper is whether beauty leads directly to the
  differences in course evaluations. Given the study design, is it
  possible to answer this question as it is phrased? If not, rephrase
  the question.
\item
  Describe the distribution of \texttt{score}. Is the distribution
  skewed? What does that tell you about how students rate courses? Is
  this what you expected to see? Why, or why not?
\item
  Excluding \texttt{score}, select two other variables and describe
  their relationship using an appropriate visualization (scatterplot,
  side-by-side boxplots, or mosaic plot).
\end{enumerate}

\hypertarget{simple-linear-regression}{%
\subsection{Simple linear regression}\label{simple-linear-regression}}

The fundamental phenomenon suggested by the study is that better looking
teachers are evaluated more favorably. Let's create a scatterplot to see
if this appears to be the case:

\begin{Shaded}
\begin{Highlighting}[]
\KeywordTok{plot}\NormalTok{(evals}\OperatorTok{$}\NormalTok{score }\OperatorTok{~}\StringTok{ }\NormalTok{evals}\OperatorTok{$}\NormalTok{bty_avg)}
\end{Highlighting}
\end{Shaded}

Before we draw conclusions about the trend, compare the number of
observations in the data frame with the approximate number of points on
the scatterplot. Is anything awry?

\begin{enumerate}
\def\labelenumi{\arabic{enumi}.}
\setcounter{enumi}{3}
\item
  Replot the scatterplot, but this time use the function
  \texttt{jitter()} on the \(y\)- or the \(x\)-coordinate. (Use
  \texttt{?jitter} to learn more.) What was misleading about the initial
  scatterplot?
\item
  Let's see if the apparent trend in the plot is something more than
  natural variation. Fit a linear model called \texttt{m\_bty} to
  predict average professor score by average beauty rating and add the
  line to your plot using \texttt{abline(m\_bty)}. Write out the
  equation for the linear model and interpret the slope. Is average
  beauty score a statistically significant predictor? Does it appear to
  be a practically significant predictor?
\item
  Use residual plots to evaluate whether the conditions of least squares
  regression are reasonable. Provide plots and comments for each one
  (see the Simple Regression Lab for a reminder of how to make these).
\end{enumerate}

\hypertarget{multiple-linear-regression}{%
\subsection{Multiple linear
regression}\label{multiple-linear-regression}}

The data set contains several variables on the beauty score of the
professor: individual ratings from each of the six students who were
asked to score the physical appearance of the professors and the average
of these six scores. Let's take a look at the relationship between one
of these scores and the average beauty score.

\begin{Shaded}
\begin{Highlighting}[]
\KeywordTok{plot}\NormalTok{(evals}\OperatorTok{$}\NormalTok{bty_avg }\OperatorTok{~}\StringTok{ }\NormalTok{evals}\OperatorTok{$}\NormalTok{bty_f1lower)}
\KeywordTok{cor}\NormalTok{(evals}\OperatorTok{$}\NormalTok{bty_avg, evals}\OperatorTok{$}\NormalTok{bty_f1lower)}
\end{Highlighting}
\end{Shaded}

As expected the relationship is quite strong - after all, the average
score is calculated using the individual scores. We can actually take a
look at the relationships between all beauty variables (columns 13
through 19) using the following command:

\begin{Shaded}
\begin{Highlighting}[]
\KeywordTok{plot}\NormalTok{(evals[,}\DecValTok{13}\OperatorTok{:}\DecValTok{19}\NormalTok{])}
\end{Highlighting}
\end{Shaded}

These variables are collinear (correlated), and adding more than one of
these variables to the model would not add much value to the model. In
this application and with these highly-correlated predictors, it is
reasonable to use the average beauty score as the single representative
of these variables.

In order to see if beauty is still a significant predictor of professor
score after we've accounted for the gender of the professor, we can add
the gender term into the model.

\begin{Shaded}
\begin{Highlighting}[]
\NormalTok{m_bty_gen <-}\StringTok{ }\KeywordTok{lm}\NormalTok{(score }\OperatorTok{~}\StringTok{ }\NormalTok{bty_avg }\OperatorTok{+}\StringTok{ }\NormalTok{gender, }\DataTypeTok{data =}\NormalTok{ evals)}
\KeywordTok{summary}\NormalTok{(m_bty_gen)}
\end{Highlighting}
\end{Shaded}

\begin{enumerate}
\def\labelenumi{\arabic{enumi}.}
\setcounter{enumi}{6}
\item
  P-values and parameter estimates should only be trusted if the
  conditions for the regression are reasonable. Verify that the
  conditions for this model are reasonable using diagnostic plots.
\item
  Is \texttt{bty\_avg} still a significant predictor of \texttt{score}?
  Has the addition of \texttt{gender} to the model changed the parameter
  estimate for \texttt{bty\_avg}?
\end{enumerate}

Note that the estimate for \texttt{gender} is now called
\texttt{gendermale}. You'll see this name change whenever you introduce
a categorical variable. The reason is that R recodes \texttt{gender}
from having the values of \texttt{female} and \texttt{male} to being an
indicator variable called \texttt{gendermale} that takes a value of
\(0\) for females and a value of \(1\) for males. (Such variables are
often referred to as ``dummy'' variables.)

As a result, for females, the parameter estimate is multiplied by zero,
leaving the intercept and slope form familiar from simple regression.

\[
  \begin{aligned}
\widehat{score} &= \hat{\beta}_0 + \hat{\beta}_1 \times bty\_avg + \hat{\beta}_2 \times (0) \\
&= \hat{\beta}_0 + \hat{\beta}_1 \times bty\_avg\end{aligned}
\]

We can plot this line and the line corresponding to males with the
following custom function.

\begin{Shaded}
\begin{Highlighting}[]
\KeywordTok{multiLines}\NormalTok{(m_bty_gen)}
\end{Highlighting}
\end{Shaded}

\begin{enumerate}
\def\labelenumi{\arabic{enumi}.}
\setcounter{enumi}{8}
\tightlist
\item
  What is the equation of the line corresponding to males? (\emph{Hint:}
  For males, the parameter estimate is multiplied by 1.) For two
  professors who received the same beauty rating, which gender tends to
  have the higher course evaluation score?
\end{enumerate}

The decision to call the indicator variable \texttt{gendermale} instead
of\texttt{genderfemale} has no deeper meaning. R simply codes the
category that comes first alphabetically as a \(0\). (You can change the
reference level of a categorical variable, which is the level that is
coded as a 0, using the\texttt{relevel} function. Use \texttt{?relevel}
to learn more.)

\begin{enumerate}
\def\labelenumi{\arabic{enumi}.}
\setcounter{enumi}{9}
\tightlist
\item
  Create a new model called \texttt{m\_bty\_rank} with \texttt{gender}
  removed and \texttt{rank} added in. How does R appear to handle
  categorical variables that have more than two levels? Note that the
  rank variable has three levels: \texttt{teaching},
  \texttt{tenure\ track}, \texttt{tenured}.
\end{enumerate}

The interpretation of the coefficients in multiple regression is
slightly different from that of simple regression. The estimate for
\texttt{bty\_avg} reflects how much higher a group of professors is
expected to score if they have a beauty rating that is one point higher
\emph{while holding all other variables constant}. In this case, that
translates into considering only professors of the same rank with
\texttt{bty\_avg} scores that are one point apart.

\hypertarget{the-search-for-the-best-model}{%
\subsection{The search for the best
model}\label{the-search-for-the-best-model}}

We will start with a full model that predicts professor score based on
rank, ethnicity, gender, language of the university where they got their
degree, age, proportion of students that filled out evaluations, class
size, course level, number of professors, number of credits, average
beauty rating, outfit, and picture color.

\begin{enumerate}
\def\labelenumi{\arabic{enumi}.}
\setcounter{enumi}{10}
\tightlist
\item
  Which variable would you expect to have the highest p-value in this
  model? Why? \emph{Hint:} Think about which variable would you expect
  to not have any association with the professor score.
\end{enumerate}

Let's run the model\ldots{}

\begin{Shaded}
\begin{Highlighting}[]
\NormalTok{m_full <-}\StringTok{ }\KeywordTok{lm}\NormalTok{(score }\OperatorTok{~}\StringTok{ }\NormalTok{rank }\OperatorTok{+}\StringTok{ }\NormalTok{ethnicity }\OperatorTok{+}\StringTok{ }\NormalTok{gender }\OperatorTok{+}\StringTok{ }\NormalTok{language }\OperatorTok{+}\StringTok{ }\NormalTok{age }\OperatorTok{+}\StringTok{ }\NormalTok{cls_perc_eval }
             \OperatorTok{+}\StringTok{ }\NormalTok{cls_students }\OperatorTok{+}\StringTok{ }\NormalTok{cls_level }\OperatorTok{+}\StringTok{ }\NormalTok{cls_profs }\OperatorTok{+}\StringTok{ }\NormalTok{cls_credits }\OperatorTok{+}\StringTok{ }\NormalTok{bty_avg }
             \OperatorTok{+}\StringTok{ }\NormalTok{pic_outfit }\OperatorTok{+}\StringTok{ }\NormalTok{pic_color, }\DataTypeTok{data =}\NormalTok{ evals)}
\KeywordTok{summary}\NormalTok{(m_full)}
\end{Highlighting}
\end{Shaded}

\begin{enumerate}
\def\labelenumi{\arabic{enumi}.}
\setcounter{enumi}{11}
\item
  Check your suspicions from the previous exercise. Include the model
  output in your response.
\item
  Interpret the coefficient associated with the ethnicity variable.
\item
  Drop the variable with the highest p-value and re-fit the model. Did
  the coefficients and significance of the other explanatory variables
  change? (One of the things that makes multiple regression interesting
  is that coefficient estimates depend on the other variables that are
  included in the model.) If not, what does this say about whether or
  not the dropped variable was collinear with the other explanatory
  variables?
\item
  Using backward-selection and p-value as the selection criterion,
  determine the best model. You do not need to show all steps in your
  answer, just the output for the final model. Also, write out the
  linear model for predicting score based on the final model you settle
  on.
\item
  Verify that the conditions for this model are reasonable using
  diagnostic plots.
\item
  The original paper describes how these data were gathered by taking a
  sample of professors from the University of Texas at Austin and
  including all courses that they have taught. Considering that each row
  represents a course, could this new information have an impact on any
  of the conditions of linear regression?
\item
  Based on your final model, describe the characteristics of a professor
  and course at University of Texas at Austin that would be associated
  with a high evaluation score.
\item
  Would you be comfortable generalizing your conclusions to apply to
  professors generally (at any university)? Why or why not?
\end{enumerate}

\end{document}
